\documentclass[bachelor, och, assignment, times]{SCWorks}
% параметр - тип обучения - одно из значений:
%    spec     - специальность
%    bachelor - бакалавриат (по умолчанию)
%    master   - магистратура
% параметр - форма обучения - одно из значений:
%    och   - очное (по умолчанию)
%    zaoch - заочное
% параметр - тип работы - одно из значений:
%    referat    - реферат
%    coursework - курсовая работа (по умолчанию)
%    diploma    - дипломная работа
%    pract      - отчет по практике
%    pract      - отчет о научно-исследовательской работе
%    autoref    - автореферат выпускной работы
%    assignment - задание на выпускную квалификационную работу
%    review     - отзыв руководителя
%    critique   - рецензия на выпускную работу
% параметр - включение шрифта
%    times    - включение шрифта Times New Roman (если установлен)
%               по умолчанию выключен
\usepackage[T2A]{fontenc}
\usepackage[cp1251]{inputenc}
\usepackage{graphicx}

\usepackage[sort,compress]{cite}
\usepackage{amsmath}
\usepackage{amssymb}
\usepackage{amsthm}
\usepackage{fancyvrb}
\usepackage{longtable}
\usepackage{array}
\usepackage[english,russian]{babel}


\usepackage[colorlinks=true]{hyperref}


\newcommand{\eqdef}{\stackrel {\rm def}{=}}

\newtheorem{lem}{Лемма}

\begin{document}

% Кафедра (в родительном падеже)
\chair{математической кибернетики и компьютерных наук}

% Тема работы
\title{Построение генератора анализаторов арифметических выражений}

% Курс
\course{4}

% Группа
\group{451}

% Факультет (в родительном падеже) (по умолчанию "факультета КНиИТ")
%\department{факультета КНиИТ}

% Специальность/направление код - наименование
%\napravlenie{02.03.02 "--- Фундаментальная информатика и информационные технологии}
%\napravlenie{02.03.01 "--- Математическое обеспечение и администрирование информационных систем}
%\napravlenie{09.03.01 "--- Информатика и вычислительная техника}
\napravlenie{09.03.04 "--- Программная инженерия}
%\napravlenie{10.05.01 "--- Компьютерная безопасность}

% Для студентки. Для работы студента следующая команда не нужна.
%\studenttitle{Студентки}

% Фамилия, имя, отчество в родительном падеже
\author{Слуцкого Алексея Дмитриевича}

% Заведующий кафедрой
\chtitle{к.\,ф.-м.\,н.} % степень, звание
\chname{С.\,В.\,Миронов}

%Научный руководитель (для реферата преподаватель проверяющий работу)
\satitle{доцент, к.\,ф.-м.\,н.} %должность, степень, звание
\saname{Г.\,Г.\,Наркайтис}

% Руководитель практики от организации (только для практики,
% для остальных типов работ не используется)
\patitle{к.\,ф.-м.\,н., доцент}
\paname{Д.\,Ю.\,Петров}

% Семестр (только для практики, для остальных
% типов работ не используется)
\term{2}

% Наименование практики (только для практики, для остальных
% типов работ не используется)
\practtype{учебная}

% Продолжительность практики (количество недель) (только для практики,
% для остальных типов работ не используется)
\duration{2}

% Даты начала и окончания практики (только для практики, для остальных
% типов работ не используется)
%\practStart{01.07.2017}
%\practFinish{14.07.2017}

% Год выполнения отчета
\date{2017}

\maketitle

Общая постановка задачи: создать калькулятор, поддерживающий заданные пользователем арифметические операции. Для решения этой задачи необходимо:
\begin{enumerate}
\item реализовать парсер правил распознавания лексем;
\item построить по заданным правилам автомат для лексического анализатора;
\item реализовать парсер заданной грамматики;
\item сгенерировать таблицы для синтаксического анализатора;
\item заданные правила трансляции преобразовать в функции на языке программирования C;
\item создать синтаксический анализатор, который взаимодействует с лексическим анализатором и осуществляет трансляцию в процессе разбора.
\end{enumerate}

В теоретической части работы необходимо описать используемые технологии и привести все необходимые предварительные сведения для практической реализации. В экспериментальной  части работы необходимо описать разработанный программный продукт, продемонстрировать примеры его тестовых запусков и сделать соответствующие выводы.

\signatureline

\end{document}
